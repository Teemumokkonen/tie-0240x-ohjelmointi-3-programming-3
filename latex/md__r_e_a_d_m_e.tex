
\begin{DoxyEnumerate}
\item Make sure you have setup ssh-\/key for your Git\-Lab account. \href{https://course-gitlab.tuni.fi/profile/keys}{\tt https\-://course-\/gitlab.\-tuni.\-fi/profile/keys}
\item Clone and add template repo as remote. Address\-: \href{mailto:git@course-gitlab.tuni.fi}{\tt git@course-\/gitlab.\-tuni.\-fi}\-:tie-\/0240x-\/ohjelmointi-\/3-\/programming-\/3-\/2020-\/2021/group\-\_\-template\-\_\-project.\-git
\item Pull from template, and do git submodule update --init in repo. Check that Course now contains something.
\item Make sure you can build the project. (Should compile without issues if your environment is setup correctly)
\end{DoxyEnumerate}

\subsection*{Submodule / Course\-Lib}

Submodule for Course\-Lib is currently configured to use ssh. If you haven't yet setup an ssh-\/key. Go do it at \href{https://course-gitlab.tuni.fi/profile/keys}{\tt https\-://course-\/gitlab.\-tuni.\-fi/profile/keys}

The page contains also instructions for generating and using existing ssh-\/keys.

Don't change anything in Course\-Lib ( You won't be able to submit changes made in it )

If you find any bugs and/or missing features you can report them in Git\-Lab \href{https://course-gitlab.tuni.fi/tie-0240x-ohjelmointi-3-programming-3-2020-2021/course/-/issues}{\tt https\-://course-\/gitlab.\-tuni.\-fi/tie-\/0240x-\/ohjelmointi-\/3-\/programming-\/3-\/2020-\/2021/course/-\//issues}

\subsection*{Doxygen documentation}

No Doxyfile is provided with the project, but you may generate it using doxygen or doxywizard. For example, \char`\"{}doxygen -\/g\char`\"{} should generate configuration file named Doxy\-File.

\subsection*{Other notes}

You should create your own code inside your own namespace \-:)

\section*{You are allowed (and probably should) make changes to this file after you have started your project. \-:)}